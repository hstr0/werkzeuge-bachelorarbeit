\documentclass{scrbook}

\input{style}
\input{macros}

% Set title and author used in the PDF meta data
\hypersetup{
  pdftitle={Quasare - Modellierung mithilfe informationstechnischer Systeme},
  pdfauthor={Henrik Strunck}
}

% Depending on which of the following two color schemes you import your thesis will be in color or grayscale. I recommend to generate a colored version as a PDF and a grayscale version for printing.

%!TEX root = bachelor_thesis.tex

% define color of example university
%\xdefinecolor{exampleuniversity}{rgb}{1, 0.5, 0}
\xdefinecolor{exampleuniversity}{rgb}{0, 0.47, 0.55}

\colorlet{maincolor}{exampleuniversity}

\colorlet{stringcolor}{green!60!black}
\colorlet{commentcolor}{black!50}
\colorlet{keywordcolor}{maincolor!80!black}

\newcommand{\imagesuffix}{-color}
%\input{schema-gray}

\newcommand{\duedate}{13. Dezember 2016}

\usepackage{caption}
\captionsetup{justification=justified}

\begin{document}
  \frontmatter
  %!TEX root = bachelor_thesis.tex

\begin{titlepage}
  \thispagestyle{empty}

  \vskip1cm

  %\pgfimage[height=2.5cm]{uni-logo-example\imagesuffix}
  \includegraphics{Imgs/uzl_logo}
  
  \vskip2.5cm
  
  \LARGE
  
  \textbf{\sffamily\color{maincolor}Quasare}

  \textit{Modellierung mithilfe informationstechnischer\\ Systeme}

  \normalfont\normalsize

  \vskip2em
  
  \textbf{\sffamily\color{maincolor}Bachelorarbeit}

  im Rahmen des Studiengangs \\
  \textbf{\sffamily\color{maincolor}Informatik} \\
  der Universität zu Lübeck

  \vskip1em

  vorgelegt von \\
  \textbf{\sffamily\color{maincolor}Henrik Strunck}

  \vskip1em
  
  ausgegeben und betreut von \\
  \textbf{\sffamily\color{maincolor}Malte Schmitz}

  \vskip1em

  mit Unterstützung von\\
  Wikipedia

  \vskip1em

  Die Arbeit ist im Rahmen des Moduls \glqq{}Werkzuge für das wissenschaftliche Arbeiten\grqq{} entstanden.

  \vfill

  Lübeck, den \duedate
\end{titlepage}

  \include{declaration}
  %!TEX root = bachelor_thesis.tex

\cleardoublepage
\thispagestyle{plain}

\pdfbookmark{Kurzfassung}{kurzfassung}
\paragraph{Kurzfassung} In der vorliegenden Arbeit werden Quasare zunächst beschrieben und physikalisch definiert. Anschließend werden Modelle entwickelt, die unter Zuhilfenahme von informationstechnischen Systemen realisiert und ausgewertet werden sollen.

\cleardoublepage
\thispagestyle{plain}

\foreignlanguage{english}{%
\pdfbookmark{Abstract}{abstract}
\paragraph{Abstract} In this thesis, quasars are first described and physically defined. Subsequently, models are developed that are to be realized and evaluated with the help of information technology systems.
}

  \cleardoublepage
  \phantomsection
  \pdfbookmark{Inhaltsverzeichnis}{tableofcontents}
  \markboth{Inhaltsverzeichnis}{}
  \tableofcontents

  \mainmatter
  
  %!TEX root = bachelor_thesis.tex

\chapter{Einleitung}

Ein Quasar ist der aktive Kern einer Galaxie, der im sichtbaren Bereich des Lichtes nahezu punktförmig (wie ein Stern) erscheint und sehr große Energiemengen in anderen Wellenlängenbereichen ausstrahlt. Er besteht aus einem Schwarzen Loch umgeben von einer Scheibe leuchtender Materie. Der Name Quasar leitet sich von \textit{quasi-stellar radio source} (‚sternartige Radioquelle‘) ab.

\begin{figure}[h]
	\centering
	\includegraphics[width=10cm]{Imgs/quasar_illustration}
	\caption{Künstlerische Darstellung eines Quasars\cite{wikiQuasar1}}
	\label{fig:darstQuasar}
\end{figure}
  %!TEX root = bachelor_thesis.tex

\chapter{Entdeckung und Namensgebung}
\label{chapter-kapitel1}

Historisch bezeichnete der Begriff kosmische Radioquellen, die in den 1950er Jahren nicht als Radiogalaxien identifiziert werden konnten, sondern in optischen Beobachtungen blau und „sternförmig“ (also nicht flächig) erschienen. 1963 stellte Maarten Schmidt fest, dass die Radioquelle 3C 273 kein naher Stern ist, sondern mit einer Rotverschiebung von 0,158 im Bereich ferner Galaxien liegt, also nur \textit{quasi} sternartig ist. Spätere Beobachtungen zeigten, dass die hellen sternartigen Quasare doch in die Kerne von Galaxien eingebettet sind, die aber wegen der großen Entfernung schwach erscheinen. Durch die starke Rotverschiebung aufgrund der Expansion des Universums wurden Quasare als sehr weit entfernte Objekte erkannt. Diese Folgerung konnte seit der Entdeckung von Gravitationslinsen unabhängig bestätigt werden. Quasare wurden inzwischen bis zu einer Rotverschiebung von 7,1 entdeckt.

\begin{figure}[h]
	\centering
	\includegraphics[width=5cm]{Imgs/QSO_APM_08279+5225_marked}
	\caption{Fotografische Aufnahme des Quasars APM08279+5225\\\hspace*{3cm}(Rotverschiebung z=3,9)\cite{wikiQuasar2}}
	\label{fig:darstQuasar}
\end{figure}

Mit der im Jahr 2010 gemachten Entdeckung, dass der 1,6 Mrd. Lichtjahre entfernte Quasar SDSS J0013+1523 als Gravitationslinse für eine 5,9 Mrd. Lichtjahre dahinter liegende Galaxie wirkt, ergibt sich eine direkte Möglichkeit zur Massenbestimmung eines Quasars.[1][2]

Die Bezeichnung \textit{QSO (quasi-stellar object)} schließt nicht nur die klassischen \textit{radiolauten} Quasare ein, sondern auch \textit{radioleise} Objekte mit schwacher Radioemission, aber sonst ähnlichen Eigenschaften. Häufig wird aber der Begriff Quasar etwas ungenau für beide Klassen benutzt.
  %!TEX root = bachelor_thesis.tex

\chapter{Physikalische Eigenschaften}
\label{chapter-kapitel2}

Da Quasare trotz ihrer großen Entfernung relativ hell erscheinen, gehören sie zu den leuchtkräftigsten Objekten im Universum. Nur sehr kurzzeitig hell aufleuchtende Phänomene (Supernova, Gammastrahlenblitz) sind möglicherweise energiereicher. Quasare sind über weite Bereiche der elektromagnetischen Strahlung hell und haben charakteristische Spektren mit sehr breiten Emissionslinien, die in rascher Bewegung befindliches Gas anzeigen.

Quasare gehören wie die schwächeren Seyfertgalaxien zur Klasse der aktiven Galaxien. Die Trennung anhand der Leuchtkraft ist rein historisch bedingt. Nach heutiger Annahme befindet sich im Zentrum aller Galaxien mit einem Bulge ein sehr massereiches Schwarzes Loch, das mehrere Millionen bis Milliarden Sonnenmassen umfassen kann. Aktive Galaxien unterscheiden sich von normalen Galaxien dadurch, dass dieses Schwarze Loch mit der Zeit an Masse zunimmt, da Materie aus der umgebenden Galaxie (interstellares Gas oder zerrissene Sterne) durch die Gravitation des Schwarzen Loches angezogen wird. Dieser Vorgang des Ansammelns von Materie wird in der Astronomie Akkretion genannt. Aufgrund der Drehimpuls­erhaltung bei der einfallenden Materie kann diese nicht direkt in das Schwarze Loch fallen, so dass sich um dieses herum eine Akkretionsscheibe bildet. Durch Reibung heizt sich diese Scheibe auf, wobei gleichzeitig Teile der Materie Drehimpuls verlieren und so in das Schwarze Loch fallen können. Die Emission der aufgeheizten Akkretionsscheibe ist das, was man als typische Strahlung des Quasars beobachtet. Sie kann eine Leuchtkraft ähnlich der von vielen Milliarden Sternen erreichen und somit mehr Licht abstrahlen als die gesamte umgebende Wirtsgalaxie. Die leuchtkräftigsten Quasare erreichen bis über $10^{14}$-fache Sonnenleuchtkraft.

Sofern die Akkretionsscheibe über ein starkes Magnetfeld verfügt, wird ein kleiner Anteil des Materiestromes in zwei Teile gerissen und in Bahnen entlang der Feldlinien des Magnetfeldes gezwungen. Anschließend werden beide Ströme senkrecht zur Ebene der Akkretionsscheibe (einer auf jeder Seite) mit relativistischer Geschwindigkeit in die umgebende Galaxie und den weiteren Weltraum abgestoßen. Diese Jets können im Radiowellen­längenbereich beobachtet werden.
  %!TEX root = bachelor_thesis.tex

\chapter{Zusammenfassung und Ausblick}
\label{chapter-fazit}

Zusammenfassend sind grundlegende Aspekte der Quasare erläutert worden und eine zukünftige Forschung in diesem Bereich weiterhin anzustreben.

  \backmatter

%  \cleardoublepage
%  \phantomsection
%  %\pdfbookmark{Abbildungsverzeichnis}{listoffigures}
%  \listoffigures
%
%  \cleardoublepage
%  \phantomsection
%  %\pdfbookmark{Tabellenverzeichnis}{listoftables}
%  \listoftables
%
%  \cleardoublepage
%  \phantomsection
%  %\pdfbookmark{Definitions- und Theoremverzeichnis}{listoftheorems}
%  \renewcommand{\listtheoremname}{Definitions- und Theoremverzeichnis}
%  \listoftheorems[ignoreall,show={Lemma,Theorem,Korollar,Definition}]

  \cleardoublepage
  \phantomsection

  \nocite{wik16, mil85, mpi16}  
  
  \pdfbookmark{Literaturverzeichnis}{bibliography}
  \bibliography{literature}
\end{document}
