%!TEX root = bachelor_thesis.tex

\chapter{Physikalische Eigenschaften}
\label{chapter-kapitel2}

Da Quasare trotz ihrer großen Entfernung relativ hell erscheinen, gehören sie zu den leuchtkräftigsten Objekten im Universum. Nur sehr kurzzeitig hell aufleuchtende Phänomene (Supernova, Gammastrahlenblitz) sind möglicherweise energiereicher. Quasare sind über weite Bereiche der elektromagnetischen Strahlung hell und haben charakteristische Spektren mit sehr breiten Emissionslinien, die in rascher Bewegung befindliches Gas anzeigen.

Quasare gehören wie die schwächeren Seyfertgalaxien zur Klasse der aktiven Galaxien. Die Trennung anhand der Leuchtkraft ist rein historisch bedingt. Nach heutiger Annahme befindet sich im Zentrum aller Galaxien mit einem Bulge ein sehr massereiches Schwarzes Loch, das mehrere Millionen bis Milliarden Sonnenmassen umfassen kann. Aktive Galaxien unterscheiden sich von normalen Galaxien dadurch, dass dieses Schwarze Loch mit der Zeit an Masse zunimmt, da Materie aus der umgebenden Galaxie (interstellares Gas oder zerrissene Sterne) durch die Gravitation des Schwarzen Loches angezogen wird. Dieser Vorgang des Ansammelns von Materie wird in der Astronomie Akkretion genannt. Aufgrund der Drehimpuls­erhaltung bei der einfallenden Materie kann diese nicht direkt in das Schwarze Loch fallen, so dass sich um dieses herum eine Akkretionsscheibe bildet. Durch Reibung heizt sich diese Scheibe auf, wobei gleichzeitig Teile der Materie Drehimpuls verlieren und so in das Schwarze Loch fallen können. Die Emission der aufgeheizten Akkretionsscheibe ist das, was man als typische Strahlung des Quasars beobachtet. Sie kann eine Leuchtkraft ähnlich der von vielen Milliarden Sternen erreichen und somit mehr Licht abstrahlen als die gesamte umgebende Wirtsgalaxie. Die leuchtkräftigsten Quasare erreichen bis über $10^{14}$-fache Sonnenleuchtkraft.

Sofern die Akkretionsscheibe über ein starkes Magnetfeld verfügt, wird ein kleiner Anteil des Materiestromes in zwei Teile gerissen und in Bahnen entlang der Feldlinien des Magnetfeldes gezwungen. Anschließend werden beide Ströme senkrecht zur Ebene der Akkretionsscheibe (einer auf jeder Seite) mit relativistischer Geschwindigkeit in die umgebende Galaxie und den weiteren Weltraum abgestoßen. Diese Jets können im Radiowellen­längenbereich beobachtet werden.